\section*{Aufgabe 1: \emph{Gammaastronomie}}

\begin{itemize}
\item[a)] 
\begin{equation*}
<N_{on}> = s - b \alpha
\end{equation*}
\item[b)] Eine Poisson--Verteilung modelliert eine Wahrscheinlichkeitsverteilung von Ereignissen, die unabhängig voneinander in eienm gewissen Zeitraum im Detektor ankommen.
\end{itemize}



\section*{Aufgabe 2: \emph{Stichprobenvarianz}}

\section*{Aufgabe 3: \emph{Maximum--Likelihood}}
\begin{itemize}
\item[a)] Bestimmung der Schätzfunktion für dir verteilung $f(x)$:
\begin{align*}
f(x)&=
	\begin{cases}
		1/b & 0 \leq x \leq b\\
		0	&\text{else}
	\end{cases}\\
L(x_1 ,... x_n,b) &= f_b(x_1)\cdot ... \cdot f_b(x_n)\\
&=\frac{1}{b^n} 1_{x_1 \in [0,b]} \cdot ... \cdot 1_{x_n \in [0,b]}\\
&=\frac{1}{b^n} 1_{x_{(n)} \leq b} \quad \text{mit: } x_{(n)} = \max\{x_i\} \\
\hat{b}(x) &= \text{argmax}_{b>0} L(b) = x_{(n)} =  \max\{x_i\} 
\end{align*}
\item[b)] Erwartungstreue der Schätzfunktion $\hat{b}(x)$:
\begin{align*}
E(\hat{b}(x)) &= \frac{1}{n} \sum_{i=1}^n \max\{x_i\}\\ 
&= \frac{n \max\{x_i\}}{n} \\
&= \max\{x_i\} = \hat{b}(x)
\end{align*}
Also ist der Schätzer Erwartungstreu.

\end{itemize}

