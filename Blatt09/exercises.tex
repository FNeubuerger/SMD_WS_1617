\section*{Aufgabe 1: \emph{Likelihoodkurve}}
\begin{itemize}
\item[a)]
\begin{align*}
L(\lambda) = \prod^3_(i=1)\frac{\lambda^{x_i}}{x_i!}e^{-\lambda}\\
=\frac{\lambda^{13}}{13!}e^{-\lambda}\frac{\lambda^{8}}{8!}e^{-\lambda}\frac{\lambda^{9}}{9!}e^{-\lambda}\\
=\frac{\lambda^{30}}{8!9!13!}e^{-3\lambda}\\
\ln{L(\lambda)} = 30\ln{\lambda}-\ln{8!9!13!}-3\lambda\\
-\ln{L(\lambda)} = -30\ln{\lambda}+\ln{8!9!13!}+3\lambda\\
\end{align*}

\item[b)]
\begin{align*}
\frac{\text{d}-\ln{L(\lambda)}}{\text{d}\lambda} = 0\\
= \frac{\text{d}(-30\ln{\lambda}+\ln{8!9!13!}+3\lambda)}{\text{d}\lambda}\\
=\frac{-30}{\lambda} + 3\\
\Rightarrow \lambda = 10\\
\end{align*}

\begin{align*}
\frac{\text{d}^2-\ln{L(\lambda)}}{\text{d}\lambda^2}|_{\lambda=10}\\
\frac{\text{d}(\sfrac{-30}{\lambda}+3}{\text{d}\lambda}|_{\lambda=10}\\
=\frac{-30}{\lambda^2}|_{\lambda=10}\\
= 0.3 \\
\end{align*}

\item[c)] Die Linien stellen verschiedene Standardabweichungen dar.

\begin{align*}
-30\ln{10}+30 + 0.5 = -30\ln{\lambda} + 3\lambda \\
\lambda_{\frac{1}{2_1}} = 8.2836\\
\lambda_{\frac{1}{2_2}} = 11.9385\\
-30\ln{10}+30 + 2 = -30\ln{\lambda} + 3\lambda \\
\lambda_{2_1} = 6.7788\\
\lambda_{2_2} = 14.1088\\
-30\ln{10}+30 + \frac{9}{2} = -30\ln{\lambda} + 3\lambda \\
\lambda_{\frac{9}{2_1}} = 5.4743\\
\lambda_{\frac{9}{2_2}} = 16.5197\\
\end{align*}

\item[d)]$-\ln{L(\lambda)} = f(\lambda)$ wird um den Punkt $\lambda  = 10 $ entwickelt.

\begin{align*}
f(\lambda) = -30\ln{\lambda}+\ln{8!9!13!}+3\lambda\\
f(10) = 6.88104\\
f'(\lambda) = 3-\frac{30}{\lambda}\\
f'(10) = 0\\
f''(\lambda) = \frac{30}{\lambda^2}\\
f''(10) = 0.3\\
T(f(\lambda),\lambda=10) = 6.88104 + 0 + \frac{0.3}{2}(\lambda-10)^2 = 0.15\lambda^2 - 3\lambda + 21.881
\end{align*}
Kürzen von $-\ln{L_{max}}$ auf beiden Seiten führt zu den Lösungen für die verschiedenen Punkte.

\begin{align*}
0.15 \lambda^2 -3\lambda + 15 = 0.5\\
\lambda_{\frac{1}{2_1}} = 8.1743\\
\lambda_{\frac{1}{2_2}} = 11.8257\\
0.15 \lambda^2 -3\lambda + 15 = 2\\
\lambda_{2_1 } = 6.3485\\
\lambda_{2_2 }= 13.6515\\ 
0.15 \lambda^2 -3\lambda + 15 = \frac{9}{2}\\
\lambda_{\frac{9}{2_1}} = 4.5228\\
\lambda_{\frac{9}{2_2}} = 15.4772\\
\end{align*}
\end{itemize}
\section*{Aufgabe 2: \emph{F-Praktikum}}

\textbf{Aufgabe 2: F-Praktikum}
\textbf{a) Designmatrix}
\\
Asymmetrie beschrieben durch
\[
f(\psi) = A_0 \cos(\psi + \delta).
\]
\\
Wir nutzen den Ansatz eines linearen Modells
\[
f(\psi) = a_1 f_1(\psi) + a_2 f_2(\psi)
\]
mit 
\[
f_1(\psi) = \cos(\psi) + f_2(\psi) = \sin(\psi).
\]
\\
Zunaechst bestimmen wir die Designmatrix $A$ mit den Spaltenelementen $f_1(\psi_i) = \cos(\psi_i)$ und $f_2(\psi_i) = \sin(\psi_i)$:
\\
A = 
$\left(
\begin{array}{cc}
1 & 0 \\
\frac{\sqrt(3)}{2} & \frac{1}{2} \\
\frac{1}{2} & \frac{\sqrt(3)}{2} \\
0 & 1 \\
-\frac{1}{2} & \frac{\sqrt(3)}{2} \\
-\frac{\sqrt(3)}{2} & \frac{1}{2} \\
-1 & 0 \\
-\frac{\sqrt(3)}{2} & -\frac{1}{2} \\
-\frac{1}{2} & -\frac{\sqrt(3)}{2} \\
0 & -1 \\
\frac{1}{2} & -\frac{\sqrt(3)}{2} \\
\frac{\sqrt(3)}{2} & -\frac{1}{2} 
\end{array}
\right)$
\\
\textbf{b) Berechnung des Vektors $a$} \\
$\vec{a} = (A^{T}A)^{-1}A^{T}\vec{y}$
\\
$A^{T}A = 
\left(
\begin{array}{cccccc}
1 & \frac{\sqrt(3)}{2} & \frac{1}{2} & 0 & -\frac{1}{2} & \cdots \\
0 & \frac{1}{2} & \frac{\sqrt(3)}{2} & 1 & \frac{\sqrt(3)}{2} & \cdots
\end{array}
\right) \cdot
\left(
\begin{array}{cc}
1 & 0 \\
\frac{\sqrt(3)}{2} & \frac{1}{2} \\
\frac{1}{2} & \frac{\sqrt(3)}{2} \\
0 & 1 \\
-\frac{1}{2} & \frac{\sqrt(3)}{2} \\
-\frac{\sqrt(3)}{2} & \frac{1}{2} \\
-1 & 0 \\
-\frac{\sqrt(3)}{2} & -\frac{1}{2} \\
-\frac{1}{2} & -\frac{\sqrt(3)}{2} \\
0 & -1 \\
\frac{1}{2} & -\frac{\sqrt(3)}{2} \\
\frac{\sqrt(3)}{2} & -\frac{1}{2} 
\end{array}
\right)
= \left(
\begin{array}{cc}
6 & 0 \\
0 & 6
\end{array}
\right)
\\
\\$
Zur Bestimmung der Inversen einer Diagonalmatrix, muss jedes diagonale Element durch sein inverses ersetzt werden.
\\
$\left(
\begin{array}{cc}
\frac{1}{6} & 0 \\
0 & \frac{1}{6}
\end{array}
\right)
\left(
\begin{array}{cccccc}
1 & \frac{\sqrt(3)}{2} & \frac{1}{2} & 0 & -\frac{1}{2} & \cdots \\
0 & \frac{1}{2} & \frac{\sqrt(3)}{2} & 1 & \frac{\sqrt(3)}{2} & \cdots
\end{array}
\right)
\\
=
\left(
\begin{array}{cccccccccccc}
\frac{1}{6} & \frac{\sqrt(3)}{12} & \frac{1}{22} & 0 & -\frac{1}{12} & -\frac{\sqrt(3)}{12} & -\frac{1}{6} & -\frac{\sqrt(3)}{12} & -\frac{1}{12} & 0 & \frac{1}{12} & \frac{\sqrt(3)}{12} \\
0 & \frac{1}{12} & \frac{\sqrt(3)}{12} & \frac{1}{6} & \frac{\sqrt(3)}{12} & \frac{1}{22} & 0 & -\frac{1}{12} & -\frac{\sqrt(3)}{12} & -\frac{1}{6} & -\frac{\sqrt(3)}{12} & -\frac{1}{12}
\end{array}
\right)$
\\
Diese Matrix wird nun mit einem Vektor y multipliziert. Dieser Vektor enthaelt die gemessenen Asymmetrien.
\\
$\vec{y}^T = 
\begin{array}{cccccccc}
( & -0,032 & 0,010 & 0,057 & 0,068 & 0,076 & 0,080 \\
0,031 & 0,005 & -0,041 & -0,090 & -0,088 & -0,074 & ) 
\end{array}$
\\
Als Loesungsvektor erhaelt man
$\vec{a} =
\left(
\begin{array}{c}
-0,0341 \\
 0,0774
\end{array}
\right)$
\begin{itemize}
\item[c)]
\begin{align*}
W(\vec{y}) = \left(0.011)^{-2}\right)\mathds{1}_{12x12}
\end{align*}

\begin{align*}
V(a)^{-1} = A^{T}WA= \left(0.011)^{-2}\right) A^{T}A = \left(0.011)^{-2}\right) \begin{pmatrix}
6 & 0\\
0 & 6
\end{pmatrix}\\
V(a) = \frac{121}{6000000}
\begin{pmatrix}
1 & 0\\
0 & 1
\end{pmatrix}\\
\end{align*}
$a_1$ und $a_2$ korrelieren nicht. Die Standartabweichung beträgt $\approx 0.00449073$ für $a_1$ und $a_2$.

\item[d)]

\begin{align*}
A_0\cos{(\psi+\delta)} = A_0(\cos{(\psi)}\cos{(\delta)}- \sin{(\psi)}\sin{(\delta)})\\
=A_0\cos{(\psi)}\cos{(\delta)}- A_0\sin{(\psi)}\sin{(\delta)}\\
=a_1\cos{(\delta)}- a_2\sin{(\delta)}\\
\cos{(\delta)} = a_1\left(\-\frac{a_2}{\sin{(\delta)}}^{-1}\right)\\
\delta = \arctan{\left(-\frac{a_1}{a_2}\right)} = 1.12\pm 0.05\\
A_0=-\frac{a_2}{\sin{\left(\arctan{\left(-\frac{a_1}{a_2}\right)}\right)}}
\end{align*}

Bestimmung der Kovarianz--Matrix

\begin{align*}
V = B^{T}V(\vec{a})B \\
B = 
\begin{pmatrix}
\frac{\partial \gamma}{\partial a_1}& \frac{\partial \gamma}{\partial a_2} \\
\frac{\partial A_0}{\partial a_1} & \frac{\partial A_0}{\partial a_2}\\
\end{pmatrix}
=
\begin{pmatrix}
\frac{a_1}{\sqrt{{a_1}²+{a_2}²}} & \frac{a_2}{\sqrt{{a_1}²+{a_2}²}} \\
\frac{a_2}{\sqrt{{a_1}²+{a_2}²}} & -\frac{a_1}{\sqrt{{a_1}²+{a_2}²}}\\
\end{pmatrix}
V = 
\begin{pmatrix}
-0.4356 & -0.9001\\
-0.9001 & 0.4356\\
\end{pmatrix}
\end{align*}
\item[e)]
Die Werte für $a_1,a_2,\delta$ und $A_0$ ändern sich nicht stark. Die Korrelation nimmt stark zu.
\end{itemize}
\section*{Aufgabe 3: \emph{Regularisierte kleinste Quadrate}}
