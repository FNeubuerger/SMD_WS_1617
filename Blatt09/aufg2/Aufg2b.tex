\documentclass{article}

\usepackage[utf8]{inputenc}
\usepackage{lmodern}
\usepackage{ngerman}
\usepackage{graphicx}

\begin{document}

\textbf{Aufgabe 2: F-Praktikum}
\textbf{a) Designmatrix}
\\
Asymmetrie beschrieben durch
\[
f(\psi) = A_0 \cos(\psi + \delta).
\]
\\
Wir nutzen den Ansatz eines linearen Modells
\[
f(\psi) = a_1 f_1(\psi) + a_2 f_2(\psi)
\]
mit 
\[
f_1(\psi) = \cos(\psi) + f_2(\psi) = \sin(\psi).
\]
\\
Zunaechst bestimmen wir die Designmatrix $A$ mit den Spaltenelementen $f_1(\psi_i) = \cos(\psi_i)$ und $f_2(\psi_i) = \sin(\psi_i)$:
\\
A = 
$\left(
\begin{array}{cc}
1 & 0 \\
\frac{\sqrt(3)}{2} & \frac{1}{2} \\
\frac{1}{2} & \frac{\sqrt(3)}{2} \\
0 & 1 \\
-\frac{1}{2} & \frac{\sqrt(3)}{2} \\
-\frac{\sqrt(3)}{2} & \frac{1}{2} \\
-1 & 0 \\
-\frac{\sqrt(3)}{2} & -\frac{1}{2} \\
-\frac{1}{2} & -\frac{\sqrt(3)}{2} \\
0 & -1 \\
\frac{1}{2} & -\frac{\sqrt(3)}{2} \\
\frac{\sqrt(3)}{2} & -\frac{1}{2} 
\end{array}
\right)$
\\
\textbf{b) Berechnung des Vektors $a$} \\
$\vec{a} = (A^{T}A)^{-1}A^{T}\vec{y}$
\\
$A^{T}A = 
\left(
\begin{array}{cccccc}
1 & \frac{\sqrt(3)}{2} & \frac{1}{2} & 0 & -\frac{1}{2} & \cdots \\
0 & \frac{1}{2} & \frac{\sqrt(3)}{2} & 1 & \frac{\sqrt(3)}{2} & \cdots
\end{array}
\right) \cdot
\left(
\begin{array}{cc}
1 & 0 \\
\frac{\sqrt(3)}{2} & \frac{1}{2} \\
\frac{1}{2} & \frac{\sqrt(3)}{2} \\
0 & 1 \\
-\frac{1}{2} & \frac{\sqrt(3)}{2} \\
-\frac{\sqrt(3)}{2} & \frac{1}{2} \\
-1 & 0 \\
-\frac{\sqrt(3)}{2} & -\frac{1}{2} \\
-\frac{1}{2} & -\frac{\sqrt(3)}{2} \\
0 & -1 \\
\frac{1}{2} & -\frac{\sqrt(3)}{2} \\
\frac{\sqrt(3)}{2} & -\frac{1}{2} 
\end{array}
\right)
= \left(
\begin{array}{cc}
6 & 0 \\
0 & 6
\end{array}
\right)
\\
\\$
Zur Bestimmung der Inversen einer Diagonalmatrix, muss jedes diagonale Element durch sein inverses ersetzt werden.
\\
$\left(
\begin{array}{cc}
\frac{1}{6} & 0 \\
0 & \frac{1}{6}
\end{array}
\right)
\left(
\begin{array}{cccccc}
1 & \frac{\sqrt(3)}{2} & \frac{1}{2} & 0 & -\frac{1}{2} & \cdots \\
0 & \frac{1}{2} & \frac{\sqrt(3)}{2} & 1 & \frac{\sqrt(3)}{2} & \cdots
\end{array}
\right)
\\
=
\left(
\begin{array}{cccccccccccc}
\frac{1}{6} & \frac{\sqrt(3)}{12} & \frac{1}{22} & 0 & -\frac{1}{12} & -\frac{\sqrt(3)}{12} & -\frac{1}{6} & -\frac{\sqrt(3)}{12} & -\frac{1}{12} & 0 & \frac{1}{12} & \frac{\sqrt(3)}{12} \\
0 & \frac{1}{12} & \frac{\sqrt(3)}{12} & \frac{1}{6} & \frac{\sqrt(3)}{12} & \frac{1}{22} & 0 & -\frac{1}{12} & -\frac{\sqrt(3)}{12} & -\frac{1}{6} & -\frac{\sqrt(3)}{12} & -\frac{1}{12}
\end{array}
\right)$
\\
Diese Matrix wird nun mit einem Vektor y multipliziert. Dieser Vektor enthaelt die gemessenen Asymmetrien.
\\
$\vec{y}^T = 
\begin{array}{cccccccc}
( & -0,032 & 0,010 & 0,057 & 0,068 & 0,076 & 0,080 \\
0,031 & 0,005 & -0,041 & -0,090 & -0,088 & -0,074 & ) 
\end{array}$
\\
Als Loesungsvektor erhaelt man
$\vec{a} =
\left(
\begin{array}{c}
-0,0341 \\
 0,0774
\end{array}
\right)$
\end{document}	