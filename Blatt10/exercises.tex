
\section*{Aufgabe 1: \emph{Gamma-Astronomie}}
\begin{itemize}
\item[a)]
\item[b)] 
\begin{equation}
\lambda = \frac{L(X|s_0 , \hat{b_0})}{L(X|\hat{s},\hat{b})} = \left( \frac{\alpha}{1+\alpha} \left( \frac{N_{on}+N_{off}}{N_{on}} \right) \right)^{N_{on}} \left( \frac{1}{1+\alpha} \left( \frac{N_{on}+N_{off}}{N_{on}} \right) \right)^{N_{off}}
\end{equation}
\item[c)]
\begin{equation}
S=\sqrt{-2\ln\lambda} = \sqrt{2}\left[ N_{on} \ln\left[\left( \frac{\alpha}{1+\alpha} \left( \frac{N_{on}+N_{off}}{N_{on}} \right) \right)\right] +  N_{off} \ln\left[\left( 1+\alpha \left( \frac{N_{on}+N_{off}}{N_{on}} \right) \right)\right] \right]^{\frac{1}{2}}
\end{equation}
\item[d)] keine zeit :(
\end{itemize}

\section*{Aufgabe 2: \emph{$\chi^2$--Test}}

\begin{align*}
\chi^2 = \left((31.6-31.3)^2+(32.2-31.3)^2+(31.2-31.3)^2+(31.9-31.3)^2+(31.3-31.3)^2+(30.8-31.3)^2+(31.3-31.3)^2\right)\frac{1}{2} = 0.38
\end{align*}

\begin{align*}
\chi^2 = \left((31.6-30.7)^2+(32.2-30.7)^2+(31.2-30.7)^2+(31.9-30.7)^2+(31.3-30.7)^2+(30.8-30.7)^2+(31.3-30.7)^2\right)\frac{1}{2} = 1.37
\end{align*}

\section*{Aufgabe 3: \emph{Likelihood-Quotienten-Test}}

\begin{itemize}
\item[a)] 
Die erwarteten Maxima der Likelihoodverteilungen liegen bei den jeweiligen Mittelwerten der Messgröße.
\begin{align*}
L(\mu_0|x) &= P(x|\mu) = \exp\left[-\frac{x-\mu_0}{2\sigma^2}\right] \\
\Gamma (x) &= \frac{L(\mu=\mu_0|x)}{\sup(L(\mu\in\Theta)|x)} = \frac{L(\mu=\mu_0|x)}{L(\mu=\mu_{\text{best}}|x)}\\
&= \exp\left[ -\frac{n}{2\sigma^2} (\bar{\mu} - \mu_0)^2 \right]\\
\text{Die Gegenhypothese trifft zu wenn: }\\
\Gamma &= \exp\left[ -\frac{n}{2\sigma^2} (\bar{\mu} - \mu_0)^2 \right] \geq k_{\text{\alpha}}\\
&\Leftrightarrow \frac{n |\bar{\mu}-\mu_0|}{\sqrt{\sigma^2}} \geq z\\
\text{mit $\alpha=0.95$ folgt $z=1.64448$}\\
\end{align*}
\item[d)] 
\begin{equation}
\frac{|205-200|}{\sqrt{\frac{10^2}{25}}} = 2.5 > z=1.64448
\end{equation}
Daraus folgt dass die Gegenhypothese zutrifft.
\end{itemize}
 
