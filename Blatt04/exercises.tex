 

\section*{Aufgabe 1: \emph{Zei Populationen}}

\section*{Aufgabe 2: \emph{Trennende Geraden}}

\begin{itemize}
\item[a]
\begin{figure}
	\centering
	\includegraphics[width=0.7\textwidth]{scatter_P0_P1.png}
	\caption{Zweidimensionaler Scatterplot der Populationen.}
\end{figure}
\begin{figure}
	\centering
	\includegraphics[width=0.7\textwidth]{scatter_with_lines.png}
	\caption{Zweidimensionaler Scatterplot der Populationen mit eingezeichneten Projektionsgeraden.}
\end{figure}
\item[b]
\begin{figure}
	\centering
	\includegraphics[width=0.7\textwidth]{hists_proj.png}
	\caption{Haeufigkeit Zufallszahlen}
\end{figure}
\item[c]





\section*{Aufgabe 3: \emph{Fisher--Diskrimanante: Per Hand}}

\section*{Aufgabe 4: \emph{Datenaufbereitung}}

\begin{itemize}

\item[1] Tokenizierung: Zerlegen eines Fließtextes in Tokens (kleine Einheiten) und anschließendes Löschen unbrauchbarer Tokens
\item[Bsp.:] Erstellung von Bbliotheken zu Natural Language Processing
\item[2] Typ und Formatierung einzelner Einträge aneinander anpassen.
\item[Bsp.:] Gleiche Formatierung für Einträge von Daten oder Zeiten.
\item[3] Ersetzen nicht eindeutiger Daten.
\item[Bsp.:] Lücken, NaNs, $\infinity$ sinnvoll ersetzen oder löschen.
\item[4] Ersetzen von missverständlichen Informationen.
\item[Bsp.:] Konstanten und fehlende Werte löschen.

\item[b] Eine Normierung sorgt für eine bessere Vergleichbarkeit von Daten.
\item[c] Löschen oder ersetzen von Daten.
\item[d] Daten müssen miteinander kombiniert werden können. Zum Beispiel anpassen von Tabellenlängen beim Zusammenführen, gleiche Formatierung von Daten.
\end{itemize}