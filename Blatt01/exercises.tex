 
\section*{Aufgabe 1: \emph{Maxwell'sche Geschwindigkeitsverteilung}}

\begin{equation}
f(v)=N\left(\frac{m}{2\pi k_BT}\right)^\frac{3}{2}\exp{\left(-\frac{mv²}{2k_bT}\right)}\cdot4\pi v²
\end{equation}

Die Normalisierungkonstante berechnet sich mit Hilfe von

\begin{equation}
\int_{-\infty}^\infty f(v)\text{d}v=1.
\end{equation}

\begin{itemize}
\item[a)] Die wahrscheinlichste Geschwindigkeit ist gegeben durch
\begin{equation}
\frac{\text{d}f(v)}{\text{d}v}=0.
\end{equation}

\item[b)] Die mittlere Geschwindigkeit ist 

\begin{equation}
\bar{v}=\int_{-\infty}^\infty v\cdot f(v)\text{d}v.
\end{equation}



\item[c)]
\item[d)]
\item[e)]
\end{itemize}
\section*{Aufgabe 3: \emph{Würfel}}
\begin{itemize}
\item[a)] Die Summe der Punkte ergibt $9$. 
\begin{align*}
P(W_{\text{r}} + W_{\text{b}} = 9) \\
 = 2P(9|W_{\text{r,b}} = 3 ) \\
 + 2P(9|W_{\text{r,b}} = 4 ) \\
 + 2P(9|W_{\text{r,b}} = 5 ) \\
 + 2P(9|W_{\text{r,b}} = 6 ) \\
 = 2/36 + 2/36 + 2/36 + 2/36 \\
 = 1/6
\end{align*}

\item[b)] Die Summe der Punkte ist $9$ oder mehr.
\begin{align*}
P(W_{\text{r}} + W_{\text{b}} > 9) \\
 = 2P(9|W_{\text{r,b}} = 4 ) \\
 + 2P(9|W_{\text{r,b}} = 5 ) \\
 + 2P(9|W_{\text{r,b}} = 6 ) \\
 = 2 \cdot 1/6 \cdot 1/6 + 2 \cdot 1/6 \cdot 1/3 + 2 \cdot 1/6 \cdot 1/2 \\
 = 1/3
\end{align*}

\item[c)] Die Wahrscheinlichkeit, dass ein Würfel $4$, der andere $5$ Punkte zeigt.
\begin{align*}
P(W_{\text{r,b}} = 5 \land W_{\text{r,b}}=4)\\
= 2P(W_{\text{r,b}} = 5) + 2P(W_{\text{r,b}} = 4)\\
= 2/3
\end{align*}

\item[d)] Die Wahrscheinlichkeit, dass der rote Würfel $4$ und 
der blaue $5$ Punkte zeigt.

\begin{align*}
P(W_{\text{r}} = 4 \land W_{\text{b}}=5)\\
= 1/3
\end{align*}

\item[e)] Die Summe der Punkte ist $9$ unter der Bedingung, dass der rote Würfel eine $4$ zeigt.
\begin{align*}
P(W_{\text{r}} + W_{\text{b}} = 9|W_{\text{r}} = 4)
= 1/6
\end{align*}


\item[f)] Die Summe der Punkte ist $9$ oder mehr unter der Bedingung, dass der rote Würfel eine $4$ zeigt.
\begin{align*}
P(W_{\text{r}} + W_{\text{b}} > 9|W_{\text{r}} = 4)
= P(W_{\text{b}} = 6|W_{\text{r}} = 4) + P(W_{\text{b}} = 5|W_{\text{r}} = 4)\\
= 1/6 + 1/6 = 2/3 
\end{align*}


\item[g)]Die Wahrscheinlichkeit, dass der rote Würfel $4$ und der blaue $5$ Punkte zeigt.
\begin{align*}
P(W_{\text{r}} = 4|W_{\text{r}} = 4) + P(W_{\text{b}} = 5|W_{\text{r}} = 4)\\
=1/6
\end{align*}
\end{itemize}

\section*{Aufgabe 4: \emph{Zweidimensionale Gaußverteilung}}

\begin{itemize}
\item[a)] Der Korellationskoeffizient berechnet sich mit
\begin{equation}
\rho(x,y)=\frac{\text{Cov}(x,y)}{\sigma_x\cdot\sigma_y}
\end{equation}
zu $\rho(x,y)=0.8.$

\item[b)] Helena schreibt auf. Siehe Vorlesung.


\end{itemize}
