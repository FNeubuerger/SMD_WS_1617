 
\section*{Aufgabe 1: \emph{Maxwell'sche Geschwindigkeitsverteilung}}

\begin{equation}
f(v)=N\left(\frac{m}{2\pi k_BT}\right)^\frac{3}{2}\exp{\left(-\frac{mv²}{2k_bT}\right)}\cdot4\pi v²
\end{equation}

Die Normalisierungkonstante berechnet sich mit Hilfe von

\begin{equation}
\int_{-\infty}^\infty f(v)\text{d}v=1.
\end{equation}

\begin{itemize}
\item[a)] Die wahrscheinlichste Geschwindigkeit ist gegeben durch
\begin{equation}
\frac{\text{d}f(v)}{\text{d}v}=0.
\end{equation}

\item[b)] Die mittlere Geschwindigkeit ist 

\begin{equation}
\bar{v}=\int_{-\infty}^\infty v\cdot f(v)\text{d}v.
\end{equation}



\item[c)]
\item[d)]
\item[e)]
\end{itemize}
\section*{Aufgabe 2: \emph{Binning}}

Die Rechnung für die Wahrscheinlichkeit der verschiedenen Ereignisse ist in Abbildung XX dargestellt.

\begin{itemize}
\item[a)] Die Summe der Punkte ergibt mit einer Wahrscheinlichkeit von XX $9$.

\item[b)] Die Summe der Punkte ergibt mit einer Wahrscheinlichkeit von XX $9$ oder mehr.

\item[c)] Die Wahrscheinlichkeit, dass ein Würfel $4$, der andere $5$ Punkte zeigt, ist XX.

\item[d)] Die Wahrscheinlichkeit, dass der rote Würfel $4$ und 
der blaue $5$ Punkte zeigt, beträgt XX.

\item[e)] Die Summe der Punkte ergibt mit einer Wahrscheinlichkeit von XX $9$.

\item[f)] Die Summe der Punkte ergibt mit einer 
Wahrscheinlichkeit von XX $9$ oder mehr.

\item[g)]Die Wahrscheinlichkeit, dass der rote Würfel $4$ und der blaue $5$ Punkte zeigt, beträgt XX.
\end{itemize}
\section*{Aufgabe 3: \emph{Würfel}}


\section*{Aufgabe 4: \emph{Zweidimensionale Gaußverteilung}}

\begin{itemize}
\item[a)] Der Korellationskoeffizient berechnet sich mit
\begin{equation}
\rho(x,y)=\frac{\text{Cov}(x,y)}{\sigma_x\cdot\sigma_y}
\end{equation}
zu $\rho(x,y)=0.8.$

\item[b)] Helena schreibt auf. Siehe Vorlesung.


\end{itemize}
