 
\section*{Aufgabe 1: \emph{Maxwell'sche Geschwindigkeitsverteilung}}

\begin{equation}
f(v)=N\left(\frac{m}{2\pi k_BT}\right)^\frac{3}{2}\exp{\left(-\frac{mv²}{2k_bT}\right)}\cdot4\pi v²
\end{equation}

Die Normalisierungkonstante berechnet sich mit Hilfe von
\begin{equation*}
\int_{0}^\infty f(v)\text{d}v=1.
\end{equation*}

\begin{align*}
\int_{0}^\infty f(v)\text{d}v= \alpha\int_{0}^\infty v²e^{-\beta v²}\text{d}v = \frac{\sqrt{\pi\alpha}}{2\beta^\frac{3}{2}}=
\sqrt{\pi^3} 2  N \sqrt{\pi^{-3}}\left(\frac{m}{2\pi k_B T}\right)^\frac{3}{2}\left(\frac{2k_B T}{m}\right)^\frac{3}{2}=N=1
\end{align*}


\begin{itemize}
\item[a)] Die wahrscheinlichste Geschwindigkeit ist gegeben durch $\sfrac{\text{d}f(v)}{\text{d}v}=0$.
\begin{align*}
\frac{\text{d}f(v)}{\text{d}v}=\alpha\frac{\text{d}}{\text{d}v}\left(e^{-\beta v²}\right)=&\alpha\left(2ve^{-\beta v²}-v³\beta e^{-\beta v²}\right)=2\alpha e^{-\beta v²}\left(v-v³\beta\right)=0\\
2\alpha v - 2\alpha v³ \beta =& 0\\
1 =& v²\beta\\
\frac{1}{\beta}=&v²\\
v=&\sqrt{\frac{2k_BT}{m}}
\end{align*}



\item[b)] Die mittlere Geschwindigkeit wird mit Hilfe der Substitution:$ v² = x, 2v\text{d}v = \text{d}x$ berechnet. 

\begin{align*}
\langle v\rangle=&\int_{0}^\infty v\cdot f(v)\text{d}v=\alpha \int_{0}^\infty v³e^{-\beta v²}\text{d}v=\frac{\alpha}{2} \int_{0}^\infty x e^{-\beta x}\text{d}x\\
= &\frac{\alpha}{2}\beta^{-2} = \frac{8N\pi {k_B}² T²}{m²}\left(\frac{m³}{2³\pi³{k_B}³T³}\right)^{\frac{1}{2}} = \sqrt{\frac{8k_B T}{\pi m}}
\end{align*}



\item[c)]
\item[d)] 
\begin{align*}
 \frac{1}{2}\sqrt{\frac{2k_BT}{m}} =&\frac{4}{\pi}\left(\frac{m}{k_BT}\right)^\frac{3}{2}e^{-\beta v²} \\\\
\frac{\sqrt{2\pi}}{8}\left(\frac{k_BT}{m}\right)² =& e^{-\beta v²}\\\\
\sqrt{\frac{\pi}{512}}{v_m}⁴ =& e^{-\beta v²}\\\\
-\ln{\left(\sqrt{\frac{\pi}{512}}{v_m}⁴\right)}\frac{2k_BT}{m}=&v²
\end{align*}
Die volle Breite auf halber Höhe ist gegeben durch
\begin{align*}
v_{FWHM}=\pm\sqrt{-\ln{\left(\sqrt{\frac{\pi}{512}}{v_m}⁴\right)}v²}.
\end{align*}
\item[e)] Die Standardabweichung ist

\begin{equation*}
\sigma_v=\langle v²\rangle-\langle v\rangle².
\end{equation*}

\begin{align*}
\langle v²\rangle=\int_{0}^\infty f(v)v²\text{d}v = \frac{2}{\sqrt{\pi}}\left(\frac{m}{2k_BT}\right)^{-1}\sqrt{\pi}\frac{13}{22} = \frac{3}{2}{v_m}²
\end{align*}
\begin{align*}
\sigma_v = \frac{3k_BT}{m} - \frac{8k_BT}{\pi m}={v_m}²\left(\frac{3}{2}-\frac{4}{\pi}\right)
\end{align*}




\end{itemize}


\section*{Aufgabe 3: \emph{Würfel}}
\begin{itemize}
\item[a)] Die Summe der Punkte ergibt $9$. 
\begin{align*}
P(W_{\text{r}} + W_{\text{b}} = 9) \\
 = 2P(9|W_{\text{r,b}} = 3 ) \\
 + 2P(9|W_{\text{r,b}} = 4 ) \\
 + 2P(9|W_{\text{r,b}} = 5 ) \\
 + 2P(9|W_{\text{r,b}} = 6 ) \\
 = 2/36 + 2/36 + 2/36 + 2/36 \\
 = 1/6
\end{align*}

\item[b)] Die Summe der Punkte ist $9$ oder mehr.
\begin{align*}
P(W_{\text{r}} + W_{\text{b}} > 9) \\
 = 2P(9|W_{\text{r,b}} = 4 ) \\
 + 2P(9|W_{\text{r,b}} = 5 ) \\
 + 2P(9|W_{\text{r,b}} = 6 ) \\
 = 2 \cdot 1/6 \cdot 1/6 + 2 \cdot 1/6 \cdot 1/3 + 2 \cdot 1/6 \cdot 1/2 \\
 = 1/3
\end{align*}

\item[c)] Die Wahrscheinlichkeit, dass ein Würfel $4$, der andere $5$ Punkte zeigt.
\begin{align*}
P(W_{\text{r,b}} = 5 \land W_{\text{r,b}}=4)\\
= 2P(W_{\text{r,b}} = 5) + 2P(W_{\text{r,b}} = 4)\\
= 2/3
\end{align*}

\item[d)] Die Wahrscheinlichkeit, dass der rote Würfel $4$ und 
der blaue $5$ Punkte zeigt.

\begin{align*}
P(W_{\text{r}} = 4 \land W_{\text{b}}=5)\\
= 1/3
\end{align*}

\item[e)] Die Summe der Punkte ist $9$ unter der Bedingung, dass der rote Würfel eine $4$ zeigt.
\begin{align*}
P(W_{\text{r}} + W_{\text{b}} = 9|W_{\text{r}} = 4)
= 1/6
\end{align*}


\item[f)] Die Summe der Punkte ist $9$ oder mehr unter der Bedingung, dass der rote Würfel eine $4$ zeigt.
\begin{align*}
P(W_{\text{r}} + W_{\text{b}} > 9|W_{\text{r}} = 4)
= P(W_{\text{b}} = 6|W_{\text{r}} = 4) + P(W_{\text{b}} = 5|W_{\text{r}} = 4)\\
= 1/6 + 1/6 = 2/3 
\end{align*}


\item[g)]Die Wahrscheinlichkeit, dass der rote Würfel $4$ und der blaue $5$ Punkte zeigt.
\begin{align*}
P(W_{\text{r}} = 4|W_{\text{r}} = 4) + P(W_{\text{b}} = 5|W_{\text{r}} = 4)\\
=1/6
\end{align*}
\end{itemize}

\section*{Aufgabe 4: \emph{Zweidimensionale Gaußverteilung}}

\begin{itemize}
\item[a)] Der Korellationskoeffizient berechnet sich mit
\begin{equation}
\rho(x,y)=\frac{\text{Cov}(x,y)}{\sigma_x\cdot\sigma_y}
\end{equation}
zu $\rho(x,y)=0.8.$

\item[b)] 
\begin{align*}
\Phi(u_1,u_2) =& ke^{-\frac{1}{2}(u_1,u_2)B
\begin{pmatrix}
 u_1\\
 u_2
\end{pmatrix}
}\\
-\frac{1}{2}(u_1,u_2)B
\begin{pmatrix}
 u_1\\
 u_2
\end{pmatrix}
= &\text{konst.}
\end{align*}

\begin{align*}
-&\frac{1}{2}(u_1,u_2)
\frac{1}{{\sigma_1}²{\sigma_2}²-{cov_x}²}
\begin{pmatrix}
{\sigma_2}^2 & -cov_x\\
-cov_x &{\sigma_1}^2 
\end{pmatrix}
\begin{pmatrix}
 u_1\\
 u_2
\end{pmatrix}
\\=&-\frac{1}{2({\sigma_1}²{\sigma_2}²-{cov_x}²)}
(u_1{\sigma_2}²-u_2cov_x-u_1cov_x+u_2{\sigma_1}²)
\begin{pmatrix}
 u_1\\
 u_2
\end{pmatrix}\\
=&\frac{1}{2({\sigma_1}²{\sigma_2}²-{cov_x}²)}
({u_1}²{\sigma_2}²-2u_1u_2cov_x+{u_2}²{\sigma_1}²)\\
=&\frac{1}{2(1-\rho²)}
({u_1}²{\sigma_2}²-2u_1u_2cov_x+{u_2}²{\sigma_1}²)\\
=&\frac{1}{2(1-\rho²)}
({u_1}²-2u_1u_2\rho+{u_2})=\text{konst.}\\
\end{align*}
mit $\rho=\frac{cov_x}{\sigma_1\sigma_2}=\frac{cov(x_1,x_2)}{\sigma_1\sigma_2}.$

Für $\frac{1}{2(1-\rho²)}({u_1}²-2u_1u_2\rho+{u_2}=1$ gilt die Ellipsengleichung
\begin{align*}
\frac{(x_1-a_1)²}{{\sigma_1}²}-2\rho\frac{(x_1-a_1)²}{{\sigma_1}²}\frac{(x_2-a_2)²}{{\sigma_2}²}+\frac{(x_2-a_2)²}{{\sigma_2}²}= 1-\rho²
\end{align*}

mit $u_i=\frac{(x_i-a_i)²}{{\sigma_i}²}$. 

\begin{figure}
\centering
\includegraphics[width=\textwidth]{plot_4b.png}
\caption{.}
\end{figure}

\item[c)]
$p_1=1.41$, $p_2=0.76$, $\alpha=\SI{0.349}{\radian}=\SI{22.02}{\degree}$.
\begin{figure}
\centering
\includegraphics[width=\textwidth]{plot_4c.png}
\caption{.}
\end{figure}
\item[d)]
\item[e)]
\begin{figure}
\centering
\includegraphics[width=\textwidth]{plot_4e.png}
\caption{t.}
\end{figure}
\item[f)]
\item[g)]
\end{itemize}
