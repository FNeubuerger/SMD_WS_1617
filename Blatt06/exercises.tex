\section*{Aufgabe 1: \emph{kNN: Implementierung}}
\begin{itemize}
\item[a)] Wenn sich Attribute sehr stark in ihren Größenordnungen unterscheiden ist der euklidische Abstand zwischen ihnen nicht sehr Aussagekräftig. Um dieses Problem zu beheben kann eine normierung der Attribute durchgeführt werden.

\item[b)] Der kNN wird als lazy-learner bezeichnet, da kein wirkliches Modell gebildet wird, sondern das Modell für jedes Testset neu berechnet wird. Dadurch sind die Laufzeiten für die "Anwendung" des Modells groß. Die "Lernphase" ist dabei allerdings kurz bzw in der Anwendungszeit enthalten. Ein Random Forest Algorithmus trainiert zunächst mit langer Laufzeit ein Modell, das dann in kurzer Zeit auf die Testdaten angewendet werden kann.

\item[c)] -> aufgabe1.py
\item[d),e),f)]
\begin{align*}
	k &=10 \\
	\mathrm{Reinheit: } &= 0.971604447974583 \\
	\mathrm{Effizienz: } &=0.9786 \\
	\mathrm{Accuracy : } 0.975 \\
	\mathrm{Signifikanz: } &= 50.36 \\
	\mathrm{Laufzeit : } &= 114.47622203826904 \\
	k &=20 \\
	\mathrm{Reinheit: } &=  0.9601822503961965 \\
	\mathrm{Effizienz: } &= 0.9694\\ 
	\mathrm{Accuracy: } &= 0.9646\\
	\mathrm{Signifikanz: } &=  50.48 \\
	\mathrm{Laufzeit: } &=136.59338474273682 \\
	\mathrm{log10(AnzahlHits)}&\\
	k &=10 \\
	\mathrm{Reinheit: } &=1.0\\
	\mathrm{Effizienz: } &=1.0\\
	\mathrm{Accuracy: } &=1.0\\
	\mathrm{Signifikanz: } &=50.0 \\
	\mathrm{Laufzeit :} &= 185.95175766944885 \\
	k &=20 \\
	\mathrm{Reinheit: }&= 1.0 \\
	\mathrm{Effizienz: }&= 0.9998 \\
	\mathrm{Accuracy: }&=  0.9999 \\
	\mathrm{Signifikanz: }&=  49.99 \\
	\mathrm{Laufzeit: }&= 186.45055532455444 \\
\end{align*}


\end{itemize}
\section*{Aufgabe 2: \emph{Lineare Klassifikation mit Softmax}}
\begin{align*}
&x_i: M \times 1&\\
&W: K \times M& \\
&b: K \times 1& \\
&f: K \times 1& \\
\end{align*}



\section*{Aufgabe 3: \emph{Binärer Entscheidungsbaum: Die erste Entscheidung}}

\begin{align*}
S(a,b,c) = -\frac{a}{b}\log_2{\frac{a}{b}}-\frac{c}{b}\log_2{\frac{c}{b}}
\end{align*}

Die Gesamtentropie berechnet sich mit Hilfe der gesamten Anzahl von Tagen, an denen Fußball gespielt beziehungsweise nicht gespielt wurde.

\begin{align*}
&S(9,14,5) = -\frac{9}{14}\log_2{\frac{9}{14}}-\frac{5}{14}\log_2{\frac{5}{14}}= 0.940\\\\
&a: \text{Anzahl aller Tage mit Fußballspiel}\\
&b: \text{Anzahl aller Spieltage}\\
&c: \text{Anzahl aller Tage ohne Fußballspiel}\\
\end{align*}

Die Information für den Schnitt auf dem Attribut \emph{Wind} berechnet sich mit Hilfe der Anzahl von Tagen, an denen das Attribut \emph{Wind} den Wert \emph{false} beziehungsweise \emph{true} aufweist und der Anzahl von Tagen, an denen innerhalb dieser Klassifikation Fußball gespielt beziehungsweise nicht gespielt wurde.

\begin{align*}
 &\text{kein Wind}: 8    &\text{Wind}:6\\
 &\text{kein Wind + kein Fußball}:2& \text{Wind + kein Fußball}:3\\
&\text{kein Wind + Fußball}:6 & \text{Wind + Fußball}:3\\
\end{align*}

\begin{align*}
&S(Wind_{false}) = -\frac{2}{8}\log_2{\frac{2}{8}}-\frac{6}{8}\log_2{\frac{6}{8}} = 0.811\\
&S(Wind_{true}) = -\frac{3}{6}\log_2{\frac{3}{6}}-\frac{3}{6}\log_2{\frac{3}{6}}= 1.000\\
&I(Wind) = S(9,14,5) - \frac{8}{14}0.81 - \frac{6}{14}1.00 = 0.048
\end{align*}


