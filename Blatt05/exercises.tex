\section*{Aufgabe 1: \emph{Fisher--Diskriminante: Implementierung}}
\begin{itemize}
\item[a)] Die Mittelwerte der Verteilungen sind: $\vec{\mu_0} = (0.0223003 ; 3.02562461) $ und $\vec{\mu_1} = (5.98570071 ; 3.09798977)$

\item[b)] Die Streumatrix innerhalb der Klassen berechnet sich zu:
\begin{equation}
S_W = 	\begin{pmatrix} 122897.81109658 & 81975.19093279\\
 						81975.19093279 & 67446.72860506
 		\end{pmatrix} 
\end{equation}
Die Streumatrix zwischen den Klassen berechnet sich zu:
\begin{equation}
S_B = 	\begin{pmatrix}	8.89053611e+04  & 1.07885616e+03 \\
						1.07885616e+03  & 1.30917934e+01
 		\end{pmatrix} 
\end{equation}
\item[c)] Die Eigenvektoren berechnen sich zu:
\begin{align*}
\vec{\lambda_1} =& ( 0.63668392 ; -0.77112488 ) \quad \text{mit:}\quad \lambda_1 = 3.71 \\
\vec{\lambda_2} =& ( -0.01213399 ; 0.99992638 ) \quad \text{mit:}\quad \lambda_2 = 1.39 \cdot 10^{-17}
\end{align*}
\item[d)] Die Projektion der Populationen $P_0$ und $P_1$ auf die Geraden $\lambda_i$.
\begin{figure}[H]
	\centering
	\includegraphics[width=0.7\textwidth]{hist_0.png}
	\caption{ Projektion der Populationen auf die Gerade $\lambda_1$}
\end{figure}
\begin{figure}[H]
	\centering
	\includegraphics[width=0.7\textwidth]{hist_1.png}
	\caption{Projektion der Populationen auf die Gerade $\lambda_2$}
\end{figure}
\item[e)] Die Performanz in Abhängigkeit des gewählten Schnittes:
\begin{figure}[H]
	\centering
	\includegraphics[width=0.7\textwidth]{performace_1.png}
	\caption{Performanz für die Projektion der Populationen auf die Gerade $\lambda_1$}
\end{figure}
\begin{figure}[H]
	\centering
	\includegraphics[width=0.7\textwidth]{performace_2.png}
	\caption{Performanz für die Projektion der Populationen auf die Gerade $\lambda_2$}
\end{figure}
\item[f)] Das Signal-Untergrundverhältnis wird für die Projektion auf $\lambda_1$ bei $\lambda_{\text{cut}}\approx -5.76$ und für die Projektion auf $\lambda_2$ bei $\lambda_{\text{cut}}\approx -5.36$ maximal.
\begin{figure}[H]
	\centering
	\includegraphics[width=0.7\textwidth]{sig_bkg_ratio.png}
	\caption{Signal-Untergrundverhätnis für die Projektion der Populationen auf die Gerade $\lambda_1$}
\end{figure}
\begin{figure}[H]
	\centering
	\includegraphics[width=0.7\textwidth]{sig_bkg_ratio2.png}
	\caption{Signal-Untergrundverhätnis für die Projektion der Populationen auf die Gerade $\lambda_2$}
\end{figure}

\item[g)] Die Signifikanz wird für die Projektion auf $\lambda_1$ bei $-10 \leq \lambda_{\text{cut}} \leq -5.96$ und für die Projektion auf $\lambda_2$ bei $-10 \leq \lambda_{\text{cut}} \leq -5.56$ maximal.
\begin{figure}[H]
	\centering
	\includegraphics[width=0.7\textwidth]{signifikanz.png}
	\caption{Signifikanz für die Projektion der Populationen auf die Gerade $\lambda_1$}
\end{figure}
\begin{figure}[H]
	\centering
	\includegraphics[width=0.7\textwidth]{signifikanz2.png}
	\caption{Signifikanz für die Projektion der Populationen auf die Gerade $\lambda_2$}
\end{figure}

\item[h)] Analog für den kleineren Datensatz von $P_0$ nur mit der Gerade $\lambda_1$.

\item[he)] Die Performanz in Abhängigkeit des gewählten Schnittes:
\begin{figure}[H]
	\centering
	\includegraphics[width=0.7\textwidth]{performace_1_h.png}
	\caption{Performanz für die Projektion der Populationen auf die Gerade $\lambda_1$}
\end{figure}

\item[hf)] Das Signal-Untergrundverhältnis wird für die Projektion auf $\lambda_1$ bei $\lambda_{\text{cut}}\approx -5.35$maximal.
\begin{figure}[H]
	\centering
	\includegraphics[width=0.7\textwidth]{sig_bkg_ratio_h.png}
	\caption{Signal-Untergrundverhätnis für die Projektion der Populationen auf die Gerade $\lambda_1$}
\end{figure}


\item[hg)] Die Signifikanz wird für die Projektion auf $\lambda_1$ bei $-10 \leq \lambda_{\text{cut}} \leq -5.55$ maximal.
\begin{figure}[H]
	\centering
	\includegraphics[width=0.7\textwidth]{signifikanz_h.png}
	\caption{Signifikanz für die Projektion der Populationen auf die Gerade $\lambda_1$}
\end{figure}



\end{itemize}
\section*{Aufgabe 2: \emph{kMeans per Hand}}

Population: $(1,4) (1,5) (1,6) (3,3) (3,2) (4,1) (5,1) (6,2) (6,3) (8,4) (8,5) (8,6)$
Clusterzentren: $(3,4) (7,4) (3,7)$
\begin{itemize}
\item[a)] 
\begin{table}
\centering
\caption{Messdaten für dubiose Elemente.}
\begin{tabular}{S S S S S S S}
\toprule
\multicolumn{1}{c}{Clusterzentrum} & \multicolumn{6}{c}{Datenpunkte} \\
\midrule
\text{(3.00, 4.00)}& \text{(1,5)} &\text{(1,4)}& \text{(3,3)}& \text{(3,2)} &\text{(4,1)}& \\
\text{(7.00, 4.00)}& \text{(6,2)} &\text{(6,3)} &\text{(8,4)} &\text{(8,5)}& \text{(8,6)} &\text{(5,1)}\\
\text{(3.00, 7.00)}& \text{(1,6)} &&&&&\\
\bottomrule
\end{tabular}
\end{table}

\begin{table}
\centering
\caption{Messdaten für dubiose Elemente.}
\begin{tabular}{S S S S S S S S}
\toprule
\multicolumn{1}{c}{Clusterzentrum} & \multicolumn{6}{c}{Datenpunkte} \\
\midrule
\text{(2.40, 3.00)}        &\text{(1,4)}& \text{(3,3)} & \text{(3,2)}& \text{(4,1)}&& \\
\text{(6.83, 3.50)}&\text{(5,1)}& \text{(6,2)}&\text{(6,3)}&\text{(8,4)}&\text{(8,5)}&\text{(8,6)} \\
\text{(1.00, 6.00) }           &\text{(1,6)}& \text{(1,5)}&&&&\\
\bottomrule
\end{tabular}
\end{table}

\begin{table}
\centering
\caption{Messdaten für dubiose Elemente.}
\begin{tabular}{S S S S S S S S}
\toprule
\multicolumn{1}{c}{Clusterzentrum} & \multicolumn{6}{c}{Datenpunkte} \\
\midrule
\text{(2.75, 2.50)}        &\text{(5,1)}& \text{(3,3)} & \text{(3,2)}& \text{(4,1)}&& \\
\text{(6.83, 3.50)}&\text{(6,2)}&\text{(6,3)}&\text{(8,4)}&\text{(8,5)}&\text{(8,6)}& \\
\text{(1.00, 5.50) }           &\text{(1,6)}& \text{(1,5)}&\text{(1,4)}&&&\\
\bottomrule
\end{tabular}
\end{table}




\begin{table}
\centering
\caption{Messdaten für dubiose Elemente.}
\begin{tabular}{S S S S S S S S}
\toprule
\multicolumn{1}{c}{Clusterzentrum} & \multicolumn{6}{c}{Datenpunkte} \\
\midrule
\text{(3.75, 1.75)}        &\text{(5,1)}& \text{(3,3)} & \text{(3,2)}& \text{(4,1)}&\text{(6,2)}& \\
\text{(7.20, 4.00)}&\text{(6,3)}&\text{(8,4)}&\text{(8,5)}&\text{(8,6)}&& \\
\text{(1.00, 5.00) }           &\text{(1,6)}& \text{(1,5)}&\text{(1,4)}&&&\\
\bottomrule
\end{tabular}
\end{table}

\begin{table}
\centering
\caption{Messdaten für dubiose Elemente.}
\begin{tabular}{S S S S S S S S}
\toprule
\multicolumn{1}{c}{Clusterzentrum} & \multicolumn{6}{c}{Datenpunkte} \\
\midrule
\text{(4.20, 1.80)}        &\text{(5,1)}& \text{(3,3)} & \text{(3,2)}& \text{(4,1)}&\text{(6,2)}& \\
\text{(7.50, 4.50)}&\text{(6,3)}&\text{(8,4)}&\text{(8,5)}&\text{(8,6)}&& \\
\text{(1.00, 5.00) }           &\text{(1,6)}& \text{(1,5)}&\text{(1,4)}&&&\\
\bottomrule
\end{tabular}
\end{table}


\begin{table}
\centering
\caption{Messdaten für dubiose Elemente.}
\begin{tabular}{S S S S S S S S}
\toprule
\multicolumn{1}{c}{Clusterzentrum} & \multicolumn{6}{c}{Datenpunkte} \\
\midrule
\text{(4.20, 1.80)}        &\text{(5,1)}& \text{(3,3)} & \text{(3,2)}& \text{(4,1)}&\text{(6,2)}& \\
\text{(7.50, 4.50)}&\text{(6,3)}&\text{(8,4)}&\text{(8,5)}&\text{(8,6)}&& \\
\text{(1.00, 5.00) }           &\text{(1,6)}& \text{(1,5)}&\text{(1,4)}&&&\\
\bottomrule
\end{tabular}
\end{table}


\end{itemize}
\section*{Aufgabe 3: \emph{Feature Selection mit dem MRMR--Algorthmus}}
\begin{itemize}
\item[a), b) und c)]
\end{itemize}
\begin{figure}[H]
	\centering
	\includegraphics[width=0.9\textwidth]{Rapidminer/Datenbereinigung.png}
	\caption{Reduktion der Datensätze auf insgesamt $4000$ Ereignisse. Anschließend werden die Daten bereinigt, indem einige Attribute, zum Beispiel wegen fehlender Werte, entfernt werden. Ein \emph{label} wird vergeben, um Signal und Untergrund voneinander unterscheiden zu können.}
\end{figure}

\begin{figure}[H]
	\centering
	\includegraphics[width=0.9\textwidth]{Rapidminer/Datenbereinigung_2.png}
	\caption{Ergebnisse der Datenbereinigung. Es sind $4000$ Ereignisse mit $146$ regulären und einem speziellen Attribut übrig. Dabei wurde unter anderem der für Aufgabe b) benötigte Operator "Remove Useless Attributes" benutzt.}
\end{figure}

\begin{figure}
	\centering
	\includegraphics[width=0.9\textwidth]{Rapidminer/Zusammenfassung.png}
	\caption{Prozess zur Erstellung eines Plots für die Stabilität der Attributsauswahl.}
\end{figure}

\begin{figure}
	\centering
	\includegraphics[width=0.9\textwidth]{Rapidminer/MRMR_results.png}
	\caption{ Projektion der Populationen auf die Gerade $\lambda_1$}
\end{figure}

\begin{figure}
	\centering
	\includegraphics[width=0.9\textwidth]{Rapidminer/Plot.png}
	\caption{Darstellung der Stabilität der zuvor getätigten Attributsauswahl.}
\end{figure}