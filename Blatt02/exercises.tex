 
\section*{Aufgabe 1: \emph{Zufallszahlen verschiedener Verteilungen}}

\begin{itemize}
\item[a)] Wahrscheinlichkeit zwischen $\frac{1}{2}$ und $\frac{1}{3}$ \\

\[
P\left(A\right) - P\left(B\right)
\]
\[
P\left(x \le \frac{1}{2}\right) - P\left(x \le \frac{1}{3}\right)
\]

Bei einer Gleichverteilung ist die summierte Wahrscheinlichkeit gleich dem Grenzwert.

\[
= \frac{1}{2} - \frac{1}{3} = \frac{3}{6} - \frac{2}{6} = \frac{1}{6}
\]
 \\

\item[b)] Wahrscheinlichkeit von $\frac{1}{2}$} \\

\[
P(A) = \frac{k}{n} = \frac{G\"unstige F\"alle} {M\"ogliche F\"alle}
\]

Wir betrachten eine Gleichverteilung im Intervall $[0,1]$ reeller Zahlen.

Günstiger Fall, Wert $\frac{1}{2}: 1$ \\
Mögliche Fälle: unendlich
\[
\frac{1}{\infty} = 0
\]
\\

\item[c)] Wahrscheinlichkeit von $\frac{1}{2}$ in einem Zufallsgenerator auf dem Computer} \\

Um die Wahrscheinlichkeit eines exakten Wertes in einem Zufallsgenerator auf einem
Computer zu bestimmen, pr\"ufen wir, ob diese Zahl auch als bin\"are Gleitkommazahl darstellbar ist.
Wir betrachten den Wert $\frac{1}{2}$, zun\"achst die erste Bin\"arstelle:

\[
 2^{-1} = 1,
\]

fuer alle weiteren Bin\"arstellen gilt

\[
2^{-2} = 2^{-3} = ... = 2^{-23} = 0
\]

Gehen wir von einer theoretischen perfekten Gleichverteilung auf einem Computer aus, 
so hat man einen g\"unstigen Fall zu $2^{23}$ m\"oglichen F\"allen.
Die Wahrscheinlichkeit $\frac{1}{2}$ zu treffen ist somit

\[
P\left(\frac{1}{2}\right) = \frac{1}{2^{23}} =\frac{1}{8388608}
\]


Nutzt man nun die Zufallsgeneratoren von Numpy oder ROOT, so sollte unter rund $10000000$ Ereignissen
 der gesuchte exakte Wert sein. Wir haben dies mit der doppelten Anzahl von Zufallszahlen 
und der Verwendung von numpy.random und ROOT.TRandom durchlaufen lassen. 
Das Ergebnis ist in Bild 1 zu sehen.
 \\
\begin{figure}[htbp]
	\centering
	\includegraphics[width=0.7\textwidth]{Gleichvert1zu2.png}
	\caption{Haeufigkeit Zufallszahlen}
\end{figure}
 \\

\item[d)] Wahrscheinlichkeit von $\frac{2}{3}$ in einem Zufallsgenerator auf dem Computer} \\

Der Wert $\frac{2}{3}$ l\"asst sich nicht exakt als bin\"are Gleitkommazahl darstellen,
somit ist es einem Zufallsgenerator mit binaeren Gleitkommazahlen nicht m\"oglich,
den Wert exakt darzustellen. Die Berechnung ist in Bild 2 zu sehen. 
Die Addition von jedem ungeraden Exponenten n\"ahert den exakten Wert an, 
jeder gerade Exponent wird \"ubersprungen, da sonst die $\frac{2}{3}$  \"uberschritten werden.

\begin{figure}[htbp]
	\centering
	\includegraphics[width=0.4\textwidth]{Binaer2zu3.png}
	\caption{Berechnung im Binaersystem}
\end{figure}




\section*{Aufgabe 2: \emph{Zufallszahlengeneratoren}}

Zufallszahlengenerator mit
\begin{equation}
x_n=(ax_{n-1}+b) \text{mod}m.
\end{equation}

\begin{itemize}


\item[a)] $a=1601$, $b=3456$, $m=10000$.
\item[b)] Das Ergebnis hängt nur vom Startwert ab, wenn dieser nicht in dem vorherigen Array (was ist damit gemeint?) enthalten ist.
\begin{figure}
\centering
\includegraphics[width=\textwidth]{linear_kongruent_random_numbers.png}
\caption{$10000$ Zufallszahlen, erzeugt mit dem in Aufgabenteil a) programmierten Zufallszahlengenerator.}
\label{fig:2b}
\end{figure}
\item[c)] In den Histogrammen sind eindeutig Strukturen (Linien oder Ebenen) zu erkennen. Dies deutet darauf hin, dass Zahlen von den vorher generierten Zahlen abhängen und nicht komplett gleichverteilt sind.
\begin{figure}
\centering
\includegraphics[width=\textwidth]{2dscatter.png}
\caption{Paare von Zufallszahlen, dargestellt als zweidimensionales Histogramm.}
\label{fig:2c1}
\end{figure}

\begin{figure}
\centering
\includegraphics[width=\textwidth]{3dscatter.png}
\caption{Tripel von Zufallszahlen, dargestellt als zweidimensionales Histogramm.}
\label{fig:2c2}
\end{figure}
\item[d)]

\item[e)] 

\begin{figure}
\centering
\includegraphics[width=\textwidth]{2dscatter_root.png}
\caption{Tripel von Zufallszahlen, dargestellt als zweidimensionales Histogramm. Im Gegensatz zu dem vorherigen zweidimensionalen Histogramm lässt sich hier keine Struktur erkennen.}
\label{fig:2e1}
\end{figure}

\begin{figure}
\centering
\includegraphics[width=\textwidth]{3dscatter_root.png}
\caption{Tripel von Zufallszahlen, dargestellt als dreidimensionales Histgramm.Im Gegensatz zu dem vorherigen dreidimensionalen Histogramm lassen sich hier keine Ebenen erkennen.}
\label{fig:2e2}
\end{figure}
\item[f)] Nur, wenn seed/m = 0.5.

\end{itemize}

\section*{Aufgabe 3: \emph{Zufallszahlengeneratoren}}


\section*{Aufgabe 4: \emph{Fehlerfortpflanzung}}
\begin{itemize}

\item[a)]

\begin{align}
y=a_0+a_1x\\
a_0=a_1=1.0\pm0.2\\
\rho = \frac{\text{cov}(a_0,a_1)}{\sigma(a_0)\sigma(a_1)}=-0.8\\
\text{cov}(a_0,a_1)=\rho \sigma(a_0)\sigma(a_1)=-0.8\cdot 0.2\cdot 0.2 = 0.032\\
\sigma_{y_1}=\sqrt{\left(\frac{\partial y}{\partial x}\sigma_x\right)²} = \sqrt{{a_1}²{\sigma_x}²}=0.2\\
\sigma_{y_2}=\sqrt{\left(\frac{\partial y}{\partial x}\sigma_x\right)²+2\frac{\partial y}{\partial x}\text{cov}(a_0,a_1)}=\sqrt{{a_1}²{sigma_x}²+2a_1\text{cov}(a_0,a_1)}=\sqrt{0.2²+1\cdot 0.032}=0.3224
\end{align}
\end{itemize}